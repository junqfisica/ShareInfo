\documentclass[a4paper, 12pt]{article}

\usepackage[portuges]{babel}
\usepackage[utf8]{inputenc}
\usepackage{amsmath}
\usepackage{indentfirst}
\usepackage{graphicx}
\usepackage{multicol,lipsum}
\usepackage{hyperref}


\newcommand*{\SignatureAndDate}[1]{%
    \par\noindent\makebox[3.5in]{\hrulefill} %\hfill\makebox[2.0in]{\hrulefill}%
    \par\noindent\makebox[2.5in][l]{#1}      %\hfill\makebox[2.0in][l]{Date}%
}%

\begin{document}
%\maketitle

\begin{titlepage}
	\begin{center}
	
	%\begin{figure}[!ht]
	%\centering
	%\includegraphics[width=2cm]{c:/ufba.jpg}
	%\end{figure}

		\Huge{Leibniz-Institut für Astrophysik Potsdam (AIP)}\\
		\large{Milky Way and the Local Volume}\\ 
		\large{CsF - Bolsas no Exterior - 9806-13-0}\\ 
		\vspace{15pt}
        \vspace{95pt}
        \textbf{\LARGE{Relatório de Atividades}}\\
		%\title{{\large{Título}}}
		\vspace{3,5cm}
	\end{center}
	
	\begin{flushleft}
		\begin{tabbing}
			Bolsista: Dr. Thiago Correr Junqueira \\
			 \\
			Supervisora: Dr. Cristina Chiappini\\
	\end{tabbing}
 \end{flushleft}
	\vspace{1cm}
	
	\begin{center}
		\vspace{\fill}
			 Outubro\\
		 2015
			\end{center}
\end{titlepage}
%%%%%%%%%%%%%%%%%%%%%%%%%%%%%%%%%%%%%%%%%%%%%%%%%%%%%%%%%%%

% % % % % % % % %FOLHA DE ROSTO % % % % % % % % % %

\begin{titlepage}
	\begin{center}
	
	%\begin{figure}[!ht]
	%\centering
	%\includegraphics[width=2cm]{c:/ufba.jpg}
	%\end{figure}

		\Huge{Leibniz-Institut für Astrophysik Potsdam (AIP)}\\
		\large{Milky Way and the Local Volume}\\ 
		\large{CsF - Bolsas no Exterior - 9806-13-0}\\ 
\vspace{15pt}
        
        \vspace{85pt}
        
		\textbf{\LARGE{Relatório Final}}
		\title{\large{Relatório Final}}
	%	\large{Modelo\\
     %   		Validação do modelo clássico}
			
	\end{center}
\vspace{1,5cm}
	
	\begin{flushright}

   \begin{list}{}{
      \setlength{\leftmargin}{4.5cm}
      \setlength{\rightmargin}{0cm}
      \setlength{\labelwidth}{0pt}
      \setlength{\labelsep}{\leftmargin}}

      \item Relatório final das atividades realizadas durante o período de dois anos como pós-doutor no instituto de astrofísica em Potsdam - Alemanha.  

      \begin{list}{}{
      \setlength{\leftmargin}{0cm}
      \setlength{\rightmargin}{0cm}
      \setlength{\labelwidth}{0pt}
      \setlength{\labelsep}{\leftmargin}}

			\item Bolsista: Dr. Thiago Correr Junqueira \
            \item Supervisora: Dr. Cristina Chiappini\

      \end{list}
   \end{list}
\end{flushright}
\vspace{1cm}
\begin{center}
		\vspace{\fill}
		 Novembro\\
		 2015
			\end{center}
\end{titlepage}
\newpage
% % % % % % % % % % % % % % % % % % % % % % % % % %
\newpage
\tableofcontents
\thispagestyle{empty}

\newpage
\pagenumbering{arabic}
% % % % % % % % % % % % % % % % % % % % % % % % % % %
\section{Resumo das atividades}

Durante esses dois anos trabalhei em projetos voltados a estrutura e evolução da Via Láctea. Tais projetos estão conectados ao projeto principal e ao interesse do grupo, ao qual estive integrado durante minha permanência em Potsdam. Abaixo, segue uma breve descrição sobre os dois principais trabalhos que realizei no instituto, sendo que o ultimo ainda esta em desenvolvimento. Publicações e apresentações dos trabalhos estão listados nas próximas seções.  

Em uma primeira etapa focamos na analise de orbitas de estrelas do disco, utilizando dados do New Catalogue of Optically Visible Open Clusters e do Apache Point Observatory Galactic (APOGUE). Desta analise conseguimos encontrar uma relação entre a variação do momento angular e da energia, que nos deu uma estimativa da velocidade angular do padrão espiral. Para mais detalhes ver; trabalhos publicados como primeiro autor em \ref{tp1}. 

Como parte do projeto principal estudamos três simulações cosmológicas, afim de analisar os fluxos verticais e radias de gás. Tais fluxos estão fortemente correlacionados ao histórico de formação estelar e por consequência com a evolução química das galáxias. A maior parte dos modelos de evolução química se baseiam em dois módulos de acreção de gás, uma acreção rápida no inicio da formação do disco, seguido de uma acreção lenta ao longo da evolução da galáxia. Esse cenário é conhecido como formação de disco galáctico de dentro para fora e deve levar a diferentes assinaturas químicas em diferentes distâncias galactocêntricas. Os resultados que obtivemos suportam tal modelo e estão em processo de submissão para um periódico científico. Em~\ref{tps} disponibilizamos um link para virtualização do trabalho em andamento.  

\section {Seminários}

Em Janeiro de 2015 fui convidado à dar um seminário, com duração de uma hora, no instituto ao qual estava associado. Maiores detalhes e o resumo do seminário podem ser encontrados no link abaixo:   
\\
\\
\url{http://www.aip.de/en/calendar/events/colloquia/thiago-junqueira-aip}

\section{Trabalhos publicados (nov/2103 - out/2015)}
\label{tp}

Baixo segue o titulo e o link para os trabalhos publicados nesses dois anos de pós-doutorado. 

\subsection{Trabalhos como primeiro autor}
\label{tp1}

Titulo:  {\bf A new method for estimating the pattern speed of spiral structure in the Milky Way}

link: \url{http://adsabs.harvard.edu/abs/2015MNRAS.449.2336J}
\\
\\
\subsection{Trabalhos como co-autor}

Titulo: {\bf The role of resonances in the evolution of galactic disks}

link: \url{http://adsabs.harvard.edu/abs/2015HiA....16..320L}
\\
\\
Titulo: {\bf The role of the corotation resonance in the secular evolution of disks of spiral galaxies}

link: \url{http://adsabs.harvard.edu/abs/2014RMxAC..44Q.180L}
\\
\\
Titulo: {\bf Bimodal chemical evolution of the Galactic disk and the Barium abundance of Cepheids}

link:\url{http://adsabs.harvard.edu/abs/2014IAUS..298...86L}

\section{Trabalho em processo de submissão}
\label{tps}
Como parte do projeto principal, estudamos o fluxo de gás utilizando três simulações cosmológicas. Entretanto, o artigo ainda está sendo desenvolvido e será enviado para publicação o mais rápido possível. Abaixo segue um link onde parte do trabalho pode ser acompanhado.
\\
\\
\url{https://www.overleaf.com/read/zpqkjxfzqxrp}


\section{Considerações Finais}

As metas iniciais propostas pelo projeto foram alcançadas com êxito. Novas ferramentas e contatos para futuras colaborações foram adquiridos. Sem mais delongas, concluímos que todas as expectativas inicias foram alcançadas. Tendo em vista que, trabalhos científicos, na área de astronomia, foram desenvolvidos e publicados em parceria entre Brasil-Alemanha.  
\\
\\
\\
\\
\hspace*{7cm} \Large{Potsdam, 30 de outubro, 2015}

\vspace{1.4in}

\SignatureAndDate{Bolsita: Dr. Thiago C. Junqueira}
\vspace{1.0in}
\SignatureAndDate{Supervisora: Dr. Cristina Chiappini}
 \\
 \\
 \\
 \\


\end{document}



